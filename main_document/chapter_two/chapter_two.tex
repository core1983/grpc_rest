\chapter{Teoretyczne podstawy integracji mikroserwisów}
\section{Wstęp}
W systemach rozproszonych zbudowanych z kilkudziesięciu lub nawet kilku tysięcy mikroserwisów komunikacja pomiędzy nimi jest jednym z najważniejszych aspektów. Od poprawnego zaprojektowania API oraz doboru właściwych technologii zależy, czy system będzie elastyczny jeśli chodzi o integrację z zewnętrznymi odbiorcami oraz wykorzysta większość zalet jakie oferują mikroserwisy.
\section{Technologie używane do komunikacji}
\subsection{REST}
Pojęcie REST (\textit{Representational State Transfer}) pojawiło się w roku 2000 w pracy doktorskiej Roya Fieldinga\cite{thomas2000architectural} i oznacza styl architektoniczny dla systemów rozproszonych. Usługa REST będąc oparta na protokole HTTP wykorzystuje udostępniane przez niego metody (\textit{GET, POST, PUT, DELETE}) do manipulacji na zdalnych zasobach do których dostęp jest wskazywany przy użyciu unikalnych wzorców identyfikatorów URI (\textit{Uniform Resource Identifier}). Metody te odwzorowują funkcje \textbf{create (utwórz)}, \textbf{read (odczytaj)}, \textbf{update (aktualizuj)} oraz \textbf{delete (usuń)} znane pod kryptonimem \textit{CRUD}.\@
W swojej pracy Fielding nie tylko przedstawił definicję usługi REST, ale także zapisał wiele wytycznych, które służyły przyszłym projektantom przy budowie swoich serwisów. Zbiór tych reguł wraz z późniejszymi pracami rozwijającymi tą dziedzinę kryje się pod pojęciem \textbf{HATEOAS (\textit{Hypermedia As The Engine Of Application State})}. Interfejs API, który jest zbudowany z wykorzystaniem zasad HATEOAS wymaga od klienta wykrycia funkcji jakie udostępnia. Klient który chciałby wykorzystać interfejs musi wykonać zapytanie GET na głównym identyfikatorze URI. W ten sposób otrzyma szczegółową listę identyfikatorów, które będą mogły zostać wykorzystane do obsługi innych operacji. Identyfikatory te nie mogą być przechowywane przez klienta wymuszając dynamiczną analizę. Wymaga to od programisty po stronie klienta zaimplementowanie mechanizmów, które będą umożliwiały dostosowanie się do zmian po stronie serwera. Reguła ta umożliwia projektantom interfejsów dowolną ich modyfikację bez obawy, że po stronie klienta wystąpią problemy z ich użytkowaniem. Pomimo faktu, iż rozwiązanie to przekłada się na lepszą skalowalność oraz rozszerzalność klientów oraz serwerów upraszczało pracę tylko projektantom po stronie serwera. Programiści po stronie klienta musieli pilnować, aby nie umieścić identyfikatora na stałe co przy rosnącym poziome skomplikowania API prowadziło do wielu pomyłek. Tym samym narodził się bardziej pragmatyczny wariant REST. Pozostawiono najlepsze cechy starszej specyfikacji wprowadzając znaczne ułatwienia dla programistów po stronie klienta.
Poniższe zasady mają za zadanie uporządkować standard REST tak, aby ułatwić integrację usług\cite{jacobson2015interfejspragmaticrest}:
\begin{itemize}
    \item działania na pojedynczych obiektach powinny korzystać z wszystkich dostępnych metod udostępnionych przez HTTP.\@ Do tej pory najczęściej wykorzystywana była metoda PUT,
    \item należy stosować kody powrotu dla odpowiedzi serwera,
    \item programista po stronie klienta powinien jakiego formatu danych ma się spodziewać wywołując metodę,
    \item każda kolejna wersja API powinna zawierać w identyfikatorze URI numer wersji,
    \item identyfikatory URI powinny być zaprojektowane według ustalonego wzorca,
    \item nagłówki HTTP powinny zawierać jedynie wymagane informacje
\end{itemize}
\begin{table}[h!]
    \begin{center}
        \caption{Przykład użycia metod HTTP wraz z opisem operacji oraz identyfikatorami URI}
        \hspace*{-1cm}
        \begin{tabular}{|M|c|L|}
            \toprule
            \textbf{Operacja}               & \textbf{Metoda} & \textbf{Identyfikator URI}                                      \\
            \midrule
            Umieszczenie produktu w koszyku & POST            & \url{http://api.v1.eshop/bucket/bucketName}                     \\
            \midrule
            Wyświetlenie zawartości koszyka & GET             & \url{http://api.v1.eshop/bucket/bucketName}                     \\
            \midrule
            Pobranie produktu z koszyka     & GET             & \url{http://api.v1.eshop/bucket/bucketName/product/productName} \\
            \midrule
            Zastępowanie produktu w koszyku & PUT             & \url{http://api.v1.eshop/bucket/bucketName/product/productName} \\
            \midrule
            Usuwanie produktu z koszyka     & DELETE          & \url{http://api.v1.eshop/bucket/bucketName/product/productName} \\
            \midrule
            Usuwanie całego koszyka         & DELETE          & \url{http://api.v1.eshop/bucket/bucketName}                     \\
            \bottomrule
        \end{tabular}
        \hspace*{-1cm}
    \end{center}
\end{table}
W swojej pierwotnej formie REST był zaprojektowany do wymiany informacji przy użyciu protokołu XML.\@ Format ten posiada olbrzymie możliwości jeśli bierze się pod uwagę tworzenie skomplikowanych obiektów z zagwarantowaną spójnością. Biorąc pod uwagę, że usługi REST charakteryzują się prostotą oraz lekkością pobieranie znacznych ilości danych w postaci XML stało się niewygodne. Godnym następcą okazał się format \textbf{JSON \textit{JavaScript Object Notation}}. Format ten został opracowany przez Douglasa Crockforda, który wykorzystał elementy języka JavaScript budując tym samym lekki oraz prosty język definicji danych. Największą zaletą formatu JSON jest możliwości translacji bezpośrednio w używanych językach programowania, bez konieczności konwersji na obiekty tak jak ma to miejsce w przypadku formatu XML.\@ Format XML wymusza również implementację mechanizmów analizowania składni ze względu na implementację atrybutów, przestrzeni nazw lub wariantów kodowania tekstu.
\begin{lstlisting}[language=xml, caption=Dane zapisane w formacie XML]
<?xml version="1.0" encoding="iso-8859-1"?>
<users>
  <user>
    <firstname>Paul</firstname>
    <surname>Sajnog</surname>
    <address>Wolczanska 12</address>
    <city>Lodz</city>
    <country>Poland</country>
    <contact>
      <phone>666 777 000</phone>
      <email>167686@edu.p.lodz.pl</email>
    </contact>
  </user>
</users>
\end{lstlisting}
\begin{lstlisting}[caption=Dane zapisane w formacie JSON]
{
    "users": {
        "user": {
            "firstname": "Paul",
            "surname": "Sajnog",
            "address": "Wolczanska 12",
            "city": "Lodz",
            "country": "Poland",
            "contact": {
                "phone": "666 777 000",
                "email": "167686@edu.p.lodz.pl"
            }
        }
    }
}
    \end{lstlisting}
\subsection{RPC}
Koncepcja RPC (\textit{Remote Procedure Call}) jest znacznie starsza niż wcześniej przedstawiona REST.\@ Technika ta polega na wywoływaniu zdalnych usług w taki sposób jakby wywoływane były lokalne metody.  